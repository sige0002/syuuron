\documentclass[a4j,11pt,onecolumn,oneside,titlepage,final]{jreport_sadasue}

% 必要なパッケージ
\usepackage[dvipdfmx]{graphicx} % 画像
\usepackage[dvipdfmx]{hyperref} % リンク
\usepackage{pxjahyper} % 日本語リンク対応
\usepackage{amsfonts, amsmath, amssymb, bm} % 数学記号やフォント
\usepackage{mathrsfs} % for \mathscr{}
\usepackage{amsthm} % 定理環境
\usepackage{here} % 画像の位置固定
\usepackage{listings} % ソースコード
\usepackage{xcolor} % ソースコードの色付け
\usepackage[margin=30truemm]{geometry}% ページレイアウトの設定(A4用紙の設定)

% 必要な情報の定義
\renewcommand{\nenndo}{6} % 年度
\renewcommand{\thesis}{修士論文} % 論文種別
\renewcommand{\belonging}{東京電機大学 大学院 未来科学研究科\\ロボット・メカトロニクス学専攻} % 所属
%\renewcommand{\studentid}{23FMR28} % 学籍番号を変えたいときはここを変更
%\renewcommand{\studentname}{貞末 祐希} % 学生氏名を変えたいときはここを変更
\renewcommand{\advisorname}{岩瀬 将美} % 指導教員名
\renewcommand{\advisor}{教授} % 指導教員役職
\renewcommand{\coadvisorname}{佐藤 康之} % 副指導教員名(新規定義)
\renewcommand{\coadvisor}{准教授} % 副指導教員役職(新規定義)

\renewcommand{\figurename}{Fig.}%初期設定では図〇と表示されるのでFig.〇に変更
\renewcommand{\tablename}{Table}%初期設定では表〇と表示されるのでTable〇に変更

% 定理環境の設定
\theoremstyle{definition}
\newtheorem{definition}{定義}[chapter]
\newtheorem{theorem}{定理}[chapter]
\newtheorem{proposition}{命題}[chapter]
\newtheorem{remark}{注意}[chapter]
\renewcommand\proofname{\bf 証明}



\lstset{% 付録に記載するソースコードのための設定
    language=C, % 言語をCに設定
    frame=single, % 四角で囲む
    numbers=left, % 行番号を左側に表示
    numberstyle=\tiny, % 行番号のスタイルを小さく
    keywordstyle=\color{blue}, % キーワードを青色に
    commentstyle=\color{green}, % コメントを緑色に
    stringstyle=\color{red}, % 文字列を赤色に
    basicstyle=\ttfamily, % フォントをタイプライタ風に
    backgroundcolor=\color{gray!10}, % 背景を薄いグレーに
    tabsize=4, % タブ幅
    breaklines=true, % 長い行を折り返し
    showstringspaces=false % 空白文字を非表示
}

% 本文
\begin{document}

% タイトルページ
\begin{titlepage}
  \begin{center}
      {\huge 令和~\nenndo~年度 }\\[10mm]
      {\LARGE \thesis}\\[20mm]
      {\LARGE ケーブル敷設に向けた単モータ駆動索状体ロボットの開発}\\[5mm]
      {\LARGE Development of Single-Actuator-Wave-Robot for cable laying}\\[60mm]
      {\LARGE \belonging}\\[10mm]
      {\LARGE \studentid~\studentname}\\[20mm]
      {\LARGE 指導教員~~\advisorname~~\advisor }
      %{\LARGE 指導教員~~\advisorname~~\advisor \\ ~~~~~~~~~~~~~~~~~~\coadvisorname~~\coadvisor}%副指導教員がいる場合はこちらを使う
  \end{center}
\end{titlepage}

\tableofcontents % 目次
\listoffigures   % 図目次
\listoftables    % 表目次

\newpage%表目次と要旨を分けるために挿入

\abstractcontent{
	ここには、論文の概要を記述してください。
}%要旨
\section{1章}
chapterでファイルを小分けにできます.
画像はこのように表示できます.
\begin{figure}[tbp]
  \centering
  \includegraphics[width=7cm]{./fig/酒井_サボり.eps}
  \vspace{-5pt}
  \caption{サンプル画像}
  \label{fig:sample}
  \end{figure}%序論
\chapter{書き方について}
chapterでファイルを小分けにできます.

\section{図の挿入}
画像の表示も可能です.
\figref{fig:sample}はサンプル画像です.
他の参照として,\Figref{fig:sample},\figsref{fig:sample}{fig:sample},\Figsref{fig:sample}{fig:sample}もあります.
figsref,Figsrefは複数の図を参照する際に使用します.
コマンドの書き方はchapter2.texを参照してください.
\begin{figure}[h]
  \centering
  \includegraphics[width=7cm]{fig/酒井_サボり.eps}
  \vspace{-5pt}
  \caption{Sample Image}
  \label{fig:sample}
  \end{figure}

\subsection{項}
項はsubsectuionで作成できます.

\newpage

\section{表の挿入}
表の表示も可能です.
\tabref{tab:sample}はサンプル表です.
他の参照として,\tabsref{tab:sample}{tab:sample}もあります.
tabsrefは複数の表を参照する際に使用します.
コマンドの書き方はchapter2.texを参照してください.

\begin{table}[htbp]
  \begin{center}
   \caption{Sample Table}
   \label{tab:sample}
   \begin{tabular}{l|c} \hline\hline % 2列のセンタリングされたテキストと罫線
    Parameter [Unit]&Variable \\ \hline % Header row
    Amplitude [deg] &$A$\\
    Phase [rad]&$P$\\
    Angular Frequency [Hz]& $F$\\\hline
   \end{tabular}
  \end{center}
 \end{table}%2章

\newpage%本文と参考文献を分けるために挿入

\bibliography{reference}%参考文献 reference.bib
\bibliographystyle{junsrt}

\acknowledgmentcontent{
本研究の遂行にあたり、指導教員の岩瀬将美教授に多大なご指導を賜りましたことを感謝申し上げます。また、研究室の仲間たちにも心より感謝いたします。(主に画像を提供してくれた酒井くん)
}%謝辞

\appendix% 付録の開始コマンド
\chapter{付録}
\section{書き方について}
付録はappendix.texに記述します.
以下にサンプルのプログラムリストを示します。

\begin{lstlisting}[language=C]
  // Hello, World!
  #include <stdio.h>
  
  int main() {
      printf("Hello, World!,\n");
      return 0;
  }
  \end{lstlisting}%付録

\end{document}
