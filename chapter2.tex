\chapter{書き方について}
chapterでファイルを小分けにできます.

\section{図の挿入}
画像の表示も可能です.
\figref{fig:sample}はサンプル画像です.
他の参照として,\Figref{fig:sample},\figsref{fig:sample}{fig:sample},\Figsref{fig:sample}{fig:sample}もあります.
figsref,Figsrefは複数の図を参照する際に使用します.
コマンドの書き方はchapter2.texを参照してください.
\begin{figure}[h]
  \centering
  \includegraphics[width=7cm]{fig/酒井_サボり.eps}
  \vspace{-5pt}
  \caption{Sample Image}
  \label{fig:sample}
  \end{figure}

\subsection{項}
項はsubsectuionで作成できます.

\newpage

\section{表の挿入}
表の表示も可能です.
\tabref{tab:sample}はサンプル表です.
他の参照として,\tabsref{tab:sample}{tab:sample}もあります.
tabsrefは複数の表を参照する際に使用します.
コマンドの書き方はchapter2.texを参照してください.

\begin{table}[htbp]
  \begin{center}
   \caption{Sample Table}
   \label{tab:sample}
   \begin{tabular}{l|c} \hline\hline % 2列のセンタリングされたテキストと罫線
    Parameter [Unit]&Variable \\ \hline % Header row
    Amplitude [deg] &$A$\\
    Phase [rad]&$P$\\
    Angular Frequency [Hz]& $F$\\\hline
   \end{tabular}
  \end{center}
 \end{table}